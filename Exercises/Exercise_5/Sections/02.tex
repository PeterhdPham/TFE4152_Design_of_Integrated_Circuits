\section{Problem 2}
\textbf{The following problems are related to the design of a Finite State Machine. The lecture notes from the 11th of October might be useful for some of the questions. (Those who did exercise 4 may find it useful to reuse parts of their solutions in this problem).}
\subsection*{a) Make a state diagram that describes a Finite State Machine ("FSM") that detects sequences of 101 on it's x input. The output, $y$, should be high if, and only if, the mentioned sequence is detected.}
\textbf{If your FSM should happen to enter any undefined state, let it return to the initial state when the earliest input is detected.}

\importimagewcaption{FSM.png}{a Finite State Machine ("FSM") that
detects sequences of 101 on it's x input.}

\subsection*{b) Show the corresponding state table.}


\begin{table}[H]
    \centering
    \caption{State table}
    \begin{tabular}{|l|l|l|l|l|l|}
        \hline
    \multicolumn{2}{|l|}{Current state} &  Input& \multicolumn{2}{|l|}{Next state} & Output \\
            $A$  &$B$& x &$A_{next}$&$B_{next}$& Y \\
            \hline
                0& 0 & 0 &     0    &    0     & 0 \\
                0& 0 & 1 &     0    &    1     & 0 \\
                0& 1 & 0 &     1    &    0     & 0 \\
                0& 1 & 1 &     0    &    1     & 0 \\
                1& 0 & 0 &     0    &    0     & 0 \\
                1& 0 & 1 &     1    &    1     & 0 \\
                1& 1 & 0 &     1    &    0     & 1 \\
                1& 1 & 1 &     0    &    1     & 1 \\
                \hline
    \end{tabular}
    \end{table}

\subsection*{c) Use one or more positive edge triggered D-flip-flops and find the flip-flop input equations, as well as a Boolean expression for the $\mathrm{y}$ output.}

$$y=AB\bar{x}+ABx=AB$$

\begin{table}[H]
\centering
\caption{A}
\begin{tabular}{llll}
    \hline
    A&B&x=0&x=1\\
    \hline
    0&0&0&0\\
    0&1&1&0\\
    1&0&0&1\\
    1&1&1&0\\
    \hline
\end{tabular}
\end{table}

\begin{table}[H]
    \centering
    \caption{B}
    \begin{tabular}{llll}
        \hline
        A&B&x=0&x=1\\
        \hline
        0&0&0&1\\
        0&1&0&1\\
        1&0&0&1\\
        1&1&0&1\\
        \hline
    \end{tabular}
    \end{table}

\subsection*{d) Show how the combinatorial logic part could be implemented in Verilog, for a Moore machine implementation (where the output is defined by the state only).}
\begin{lstlisting}
module FSM(
    input A;
    input B;
    output y;

);
and g1(y, A, B)
end module
\end{lstlisting}

\subsection*{e) Estimate the total number of transistors that would be needed to implement your combinatorial logic part of the FSM from the previous problem, assuming that you are 
restricted to basic Boolean gates (NAND, NOR, INVERT, XOR et cetera).}
\importimagewcaptionw{transistor_FSM.png}{AND gate using transistors}{0.3}

Using the design in figure \ref*{fig:transistor_FSM.png}, we need a total of 4 transistors to implement the combinatorial logic part of the FSM

\subsection*{f) Using the same restrictions as for the previous question, try and see if you can find an implementation that uses fewer transistors. What's the total number now?}

No, my design is perfect.

\subsection*{g) Are you aware of any logic implementation style other than static CMOS that might have been used for the implementation of the combinatorial logic?}

no, not yet. I assume that there is given this question.

\subsection*{h) Choose one alternative logic implementation style and mention one potential positive side of the alternative, and one negative.}

you could implement it using a NAND and a NOT gate to implement the same, which would be worse as it would require more transistors (6) (using CMOS)
\subsection*{i) Explain how you would like to simulate the FSM, and verify it's functionality. How should the input signal and the clock signal be related, for example?}

I would simulate it in a simple logic gate simulator on web as this isnt the most complex FSM. I would expect the FSM to trigger whenever clock goes from low to high.