\section{Problem 1}
From the problem text, we are given this information:

\begin{table}[h]
    \centering
    \caption{NMOS Transistor Parameters in 180 nm Technology}
    \begin{tabular}{ccc}
        \hline
        Parameter & Unit/Equation & Value \\
        \hline
        \( V_{\text{eff}} \) & \( \text{V} \) & \( 0.2 \) \\
        \( V_{\text{Drain}} \) & \( \text{V} \) & \( 0.2 \) \\
        \( V_{\text{Source}} \) & \( \text{V} \) & \( 0 \) \\
        \( W \) & \( \mu\text{m} \) & \( 0.5 \) \\
        \( L \) & \( \mu\text{m} \) & \( 0.2 \) \\
        \( T \) & \( K \) & \( 293 \) \\
        \( \mu C_{OX} \) & \( \frac{\mu \text{A}}{\text{V}^2} \) & \( 270 \) \\
        \( V_{t0} \) & \( \text{V} \) & \( 0.45 \) \\
        \( \lambda \cdot L \) & \( \frac{\mu\text{m}}{\text{V}} \) & \( 0.08 \) \\
        \( C_{OX} \) & \( \frac{\text{fF}}{\mu\text{m}^2} \) & \( 8.5 \) \\
        \( t_{OX} \) & \( \text{nm} \) & \( 5 \) \\
        \( n \) & \( - \) & \( 1.6 \) \\
        \( \theta \) & \( \frac{1}{\text{V}} \) & \( 1.7 \) \\
        \( m \) & \( - \) & \( 1.6 \) \\
        \( \frac{C_{OV}}{W} = L_{OV} C_{OX} \) & \( \frac{\text{fF}}{\mu\text{m}} \) & \( 0.35 \) \\
        \( \frac{C_{db}}{W} \approx \frac{C_{sb}}{W} \) & \( \frac{\text{fF}}{\mu\text{m}} \) & \( 0.5 \) \\
        \hline
    \end{tabular}
\end{table}



\subsection*{a) What can you say about the region of operation for the transistor, based on the description above?}

The region of operation where $\text{V}_\text{DS}>\text{V}_\text{DS-sat}$, the drain current is in is independend of $\text{V}_\text{DS}$ and is called the active region: 

\importimagewcaption{region_of_operation}{The \( I_{D} \) versus \( V_{D S} \) curve for an ideal MOS transistor. For \( \mathrm{V}_{\mathrm{DS}}>\mathrm{V}_{\mathrm{DS} \text {-sat, }} \mathrm{I}_{\mathrm{D}} \) is approximately constant.}

Based on the paramters in the table and figure above we can say that the region of operation is when $\text{V}_\text{DS}>0.2\text{V}$ as $V_{\text {DS-sat }}$ is given by

\begin{equation*}
    V_{\text {DS-sat }}=V_{G S}-V_{t n}=V_{\text {eff }} 
\end{equation*}

\subsection*{b) Consider a similar case as described above. except that \( V_{D S} \) is increased by \( 50 \mathrm{mV} \). Can you estimate the resulting drain current, \( I_{D S} \)?}
From figure \ref{fig:region_of_operation}, we can se that the current $I_D$ changes marginaly in the active region when $V_DS$ increases, therfore the resulting drain current can be estimated to be  

\begin{equation*}
    I_{D}=\frac{2.7\cdot10^{-4}}{2} \frac{5\cdot10^{-5}}{2\cdot10^{-5}}\left(0.2\right)^{2}=13.5\mu A
\end{equation*}

\subsection*{c) Estimate the maximum value for $V_{\text{eff}}$ given that you would avoid mobility degradation for this transistor.}
Due to mobility degadation, increases in $V_{\text{eff}}$ beyond $\frac{1}{2\theta}$:
\begin{itemize}
    \item fail to provide significant increases in small-signal transconductance,
    \item reduce the available signal swing limited by the fixed supply voltages, and
    \item reduce transistor intrinsic gain \textbf{$A_i$} dramatically
\end{itemize}

Therefore the maximum value for $V_{\text{eff}}$ given that i would avoid mobility degradation for this transistor is 

\begin{align*}
    V_{\text{eff}}&<\frac{1}{2\cdot1.7}\\
    V_{\text{eff}}&<\frac{5}{17}V
\end{align*}


\subsection*{d) Calculate the small signal output resistance for the transistor when it's operating in the active region, and square-law behaviour holds.}

We can use this equation

\begin{equation*}
    r_{o}=\frac{1}{\lambda \cdot I_{D}}
\end{equation*}

to find the small signal output resistance using the current $I_D$ found in b)

\begin{equation*}
    r_{o}=\frac{1}{0.4 \frac{1}{V} \cdot 13.5\mu A}=185.2k \Omega
\end{equation*}

\subsection*{e) What would be the small-signal output resistance if the \( \mathrm{W} / \mathrm{L} \) ratio was kept, and the gate length multiplied by 3?}

By keeping the \( \mathrm{W} / \mathrm{L} \) ratio, we know that the current $I_D$ will stay the same,

\begin{equation*}
    \Lambda=\frac{0.08}{0.6}=\frac{2}{15}\frac{1}{V}
\end{equation*}

This gives us
\begin{equation*}
    r_{o}=\frac{1}{\frac{2}{15} \frac{1}{V} \cdot 13.5\mu A}=555.6k \Omega
\end{equation*}

\subsection*{f) Calculate the small-signal transconductance, \( g_{m} \), for the transistor under the conditions relevant for problem b), above.}

The small-signal transconductance, \( g_{m} \) is given by 

\begin{equation*}
    \frac{2 \mathrm{I}_{\mathrm{D}}}{\mathrm{V}_{\mathrm{eff}}}=\mu_{\mathrm{n}} \mathrm{C}_{\mathrm{ox}} \frac{\mathrm{W}}{\mathrm{L}} \mathrm{V}_{\mathrm{eff}}
\end{equation*}

\begin{equation*}
    g_m=\frac{2 \mathrm{I}_{\mathrm{D}}}{\mathrm{V}_{\mathrm{eff}}}=\frac{2\cdot 13.5\mu A}{0.2V}=1.35\cdot10^{-4}
\end{equation*}

\subsection*{g) Indicate roughly the value for \( g_{s} \) and draw the low-frequency small-signal schematics for the transistor in the active region, and include relevant values based on what you have been asked to find, this far. Use \( r_{\text {out }} \) from \( 1 \mathrm{~d}) \).}

\importimage{smallsignal.png}


\subsection*{h) Problem 1 c) relates to mobility degradation for relatively high values of \( V_{D S} \). Can you point to any other potential problems that could emerge from high \( V_{D S} \) voltages, especially for the shortest gate-lengths?}




\subsection*{i) Make a sketch of the small-signal diagram from \( 1 \mathrm{~g} \) ), where you include the parasitic capacitances.}


\subsection*{j) Consider the layout in Figure 2.14 in CJM and an implementation in the \( 0.8 \mu \mathrm{m} \) process from Table \( 1.5 \mathrm{in} \) CJM for your calculations, and Let \( C_{j s} \) have a value of nfF in your expression. Find an expression for the smallsignal capacitance between the source and bulk nodes.}


\subsection*{k) Estimate the minimum \( V_{G S} \) required for this NMOS transistor to be in the active region.}


\subsection*{l) Circuits may in some cases operate with a supply voltage below the absolute values of both the pmos and nmos threshold voltages. Could you explain about when this could be beneficial for analog or digital circuits, or both?}