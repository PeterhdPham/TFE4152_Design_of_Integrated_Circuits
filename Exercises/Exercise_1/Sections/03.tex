\section{Problem 3}

The pn-junction in Figure \ref*{fig:pn-junction} was fabricated using a $100 \mu m$ x $100 \mu m$ mask. The pnjunction is backward biased with a 5 V voltage source, as is shown in the Figure \ref*{fig:pn-junction}. Due to thermal generation, $3.2 \cdot 10^7$ new electron-hole pairs are generated each second in the depletion region of the diode.

\importimagewcaption{pn-junction}{A pn-junction in a p-type substrate is reverse biased with a 5 V voltage source.}

\subsection*{a) How much reverse leakage current will flow in the diode? (HINT: The charge of an
electron is $q = 1.6 \cdot 10^{-19} C$)}

We wan use the formula

\begin{equation*}
    I=q\cdot n \cdot A
\end{equation*}

Where $I$ is the current, $q$ is the charge of electron, $n$ is the number of electropn-hole pairs generated per second and A is the cross-sectional area of the diode. Given that $3.2 \cdot 10^7$ new electron-hole pairs are generated each second in the depletion region of the diode, this equals to 

\begin{equation*}
    I=1.6 \cdot 10^{-19} C \cdot 3.2 \cdot 10^7 s^{-1} \cdot 100 \mu m \cdot 100 \mu m = 5.12\cdot 10^{-20} A
\end{equation*}

\subsection*{b) Does this current depend on the reverse bias voltage? Why?}
This current does not depend on the reverse bias voltage as the current is determined by the the thermal generation, which is primarily affected by the temperature.

\subsection*{c) Does this current increase or decrease if the temperature increases? Why?}
As answered in problem 3b) the current is affected by the increase or decrease of the temperature, this is because the number of carriers approximately doubles for every $11^o C$ increase in temperature. This means that a increased temperature will result in an increased thermal generation.

\subsection*{d) Next to this diode, another similar diode was fabricated using a $200 \mu m$ x $200 \mu m$
mask. How much is the reverse leakage current in this diode, if the temperature
and the bias voltage are the same as with the $200 \mu m$ x $200 \mu m$ diode?}

Using the same equation as in problem 3a) we get

\begin{equation*}
    I=1.6 \cdot 10^{-19} C \cdot 3.2 \cdot 10^7 s^{-1} \cdot 200 \mu m \cdot 200 \mu m = 2.048\cdot 10^{-19} A
\end{equation*}

