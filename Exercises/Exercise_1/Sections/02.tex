\section{Problem 2}
A piece of Silicon is doped with arsenic at a concentration of $4 \cdot 10^25 \frac{atoms}{m^3}$

\subsection*{a) Is the material n-type or p-type?}

This material is called n-type dopants since the free carriers resulting from their use have negative vharge.

\subsection*{b) The temperature is $300 K$ and the intrinsic carrier concentration $n_i$ at this temperature is $1.1 \cdot 10^16 \frac{carriers}{m^3}$. Estimate the carrier concentrations.}

Since we have an n-type impurity used, the total number of negative carriers or electrons is almost the same as the doping concentration, and is much greater than the number of free electrons in intrinsic silicon. In other words,

\begin{equation*}
    n_n = N_D
\end{equation*}

where $n_n$ denotes the free-electron concentration in n-type material and $N_D$ is the doping concentration. We can therefore estimate that the carrier concentration will be approximately $4 \cdot 10^{25} \frac{atoms}{m^3}$

\subsection*{c) Estimate the electron and hole concentrations in the same material at a temperature
of $322 K$.}

The number of carriers approximately doubles for every $11 ^o C$ increase in temperature.

Therefore we will get that the electron and hole concentrationw in the same material at a temperature of $322 K$ is equal total to 

\begin{equation*}
    4 \cdot 10^{25} \frac{atoms}{m^3} \cdot 2^2 = 1.6 \cdot 10^ {25}\frac{atoms}{m^3}
\end{equation*}

\subsection*{d) How does a change in temperature change the conductivity $\sigma$ of:}
\begin{enumerate}
    \item Intrinsic silicon?
    \subitem An increased tempeature will increase the conductivity, while a decreased temperature will have the opposite result.
    \item Extrinsic (heavily doped) silicon?
    \subitem  With the extrinsic silicon the the carrier concentration will increace linear at low temperatures, and then increase exponentialy at high temperatures as the ammount of holes/electrons is given by:
    
    \begin{align*}
        p_n&=\frac{n_i ^2}{N_D} \\
        n_p&=\frac{n_i ^2}{N_A}
    \end{align*}

    The higher the carrier concentration, the higher the conductivity will be.
\end{enumerate}



